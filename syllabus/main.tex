\documentclass[a4paper,12pt,bold]{scrartcl}

\usepackage{color,colortbl}																			%Farbige Tabellen
\usepackage{xcolor}

\usepackage{apacite}
\usepackage{footnote}
\makesavenoteenv{tabular}
\usepackage{enumitem}

\renewcommand{\baselinestretch}{1.3}\normalsize
\newcommand{\vect}[1]{\mathbf{#1}}
\newcommand{\thin}{\thinspace}
\newcommand{\thick}{\thickspace}
\newcommand{\N}{\mathcal{N}}	%Normal Distribution
\newcommand{\U}{\mathrm{U}}	%Uniform Distribution
\newcommand{\D}{\mathrm{D}}	%Dirichlet Distribution
\newcommand{\W}{\mathrm{W}}	%Wishart Distribution
\newcommand{\E}{\mathrm{E}}		%Expectation
\newcommand{\Ind}{\mathbb{I}\,}	%Indicator Function

\newcommand{\TR}{\mathrm{TR}\,}
\newcommand{\OV}{\mathrm{OV}\,}

\newcommand{\bs}{\boldsymbol}
\newcommand{\var}{\mathrm{var}\thin}
\newcommand{\plim}{\mathrm{plim}\thin}
\newcommand{\cov}{\mathrm{cov}\thin}
\newcommand\indep{\protect\mathpalette{\protect\independenT}{\perp}}
\def\independenT#1#2{\mathrel{\rlap{$#1#2$}\mkern5mu{#1#2}}}
\usepackage{bbm}
%\usepackage{endfloat}
\renewcommand{\vec}[1]{\mathbf{#1}}

\usepackage{algpseudocode,tabularx,ragged2e}
\newcolumntype{C}{>{\centering\arraybackslash}X} % centered "X" column
\newcolumntype{L}{>{\arraybackslash}X} % centered "X" column

\usepackage{algorithmicx}

\usepackage{algorithm}

\let\Algorithm\algorithm
\renewcommand\algorithm[1][]{\Algorithm[#1]\setstretch{1.5}}


\definecolor{lightgrey}{gray}{0.90}	%Farben mischen
\definecolor{grey}{gray}{0.85}
\definecolor{darkgrey}{gray}{0.65}
\definecolor{lightblue}{rgb}{0.8,0.85,1}

\newcolumntype{g}{>{\columncolor{gray}}c}
\usepackage{bibentry}
\usepackage{booktabs}
\usepackage{epigraph}
\usepackage[sans]{dsfont}
\usepackage[round,longnamesfirst]{natbib}
\usepackage{bm}																									%matrix symbol
\usepackage{setspace}																						%Fu�noten (allgm.
\usepackage[colorlinks = true,
            linkcolor = blue,
            urlcolor  = blue,
            citecolor = blue,
            anchorcolor = blue]{hyperref}%Zeilenabst�nde)
\usepackage{threeparttable}
\usepackage{lscape}																							%Querformat
\usepackage[latin1]{inputenc}																		%Umlaute
\usepackage{graphicx}
\graphicspath{{material/}}

\usepackage{amsmath, placeins}
\usepackage{amssymb}
\usepackage{fancybox}																						%Boxen und Rahmen
\usepackage{appendix}
\usepackage{listings}
\usepackage{xr}

\usepackage{enumerate}


%\usepackage{lineno}
%\linenumbers

        																%EURO Symbol
\usepackage{tabularx}
\usepackage{longtable,tabu}
\usepackage{subfig,float}																				%Mehrseitige Tabellen
\usepackage{color,colortbl}																			%Farbige Tabellen
\usepackage[left=2cm, right=2cm, top=2cm, bottom=2.5cm]{geometry} %Seitenr�nder
%\usepackage[normal]{caption2}[2002/08/03]												%Titel ohne float - Umgebung
\definecolor{lightgrey}{gray}{0.95}	%Farben mischen
\definecolor{grey}{gray}{0.85}
\definecolor{darkgrey}{gray}{0.80}

\newcommand{\mc}{\multicolumn}

\usepackage{tikz}
\usetikzlibrary{positioning}

\usepackage[labelfont=bf]{caption}
\captionsetup[table]{skip=10pt}

\usepackage{url}  % Used for linebreaks in verbatim statements

\newtheorem{Definition}{Definition}
\newtheorem{Remark}{Remark}
\newtheorem{Lemma}{Lemma}
\newtheorem{Theorem}{Theorem}
\newtheorem{Excercise}{Excercise}
\newtheorem{Result}{Result}
\newtheorem{Proposition}{Proposition}
\newtheorem{Prediction}{Prediction}
\newtheorem{Solution}{Solution}
\newtheorem{Problem}{Problem}

\setlength{\skip\footins}{1.0cm}
\deffootnote[1em]{1.1em}{0em}{\textsuperscript{\thefootnotemark}}
\renewcommand{\arraystretch}{1.05}

\DeclareMathOperator*{\argmin}{arg\,min}
\DeclareMathOperator*{\argmax}{arg\,max}

\makeatletter
\newenvironment{manquotation}[2][2em]
  {\setlength{\@tempdima}{#1}%
   \def\chapquote@author{#2}%
   \parshape 1 \@tempdima \dimexpr\textwidth-2\@tempdima\relax%
   \itshape}
  {\par\normalfont\hfill--\ \chapquote@author\hspace*{\@tempdima}\par\bigskip}
\makeatother

\newenvironment{boenumerate}
{\begin{enumerate}\renewcommand\labelenumi{\textbf{(\theenumi)}}}
{\end{enumerate}}


% !TEX root = ../main.tex
\title{Introduction to scientific computing}
\author{OpenSourceEconomics\thanks{Corresponding author: Philipp Eisenhauer, peisenha@uni-bonn.de.}}
\date{\today}


\begin{document}

\nobibliography*

% !TEX root = ../main.tex

\maketitle



\setcounter{page}{1}
\thispagestyle{empty}

\vspace{0.5cm}\begin{abstract}\noindent
We teach basic software engineering, numerical methods, and computational engineering skills. They allow you to leverage tools from computational science and increase the transparency and extensibility of our implementations. In doing so, you expand the set of possible economic questions that you can address and improve the quality of your answers.
\end{abstract}


\noindent We organize the course around the two flagship codes of our group. We use two codes to explore selected issues in software engineering, numerical methods, and computational engineering.

\begin{itemize}

\item \verb+respy+

We maintain a \verb+Python+ package for the simulation and estimation of a prototypical finite-horizon dynamic discrete choice model based on \citet{Keane.1997}.\vspace{0.3cm}

\begin{tabular}{ll}
\textbf{GitHub}		& \url{OpenSourceEconomics/respy}\\
\textbf{Docs}     & \url{respy.readthedocs.org}
\end{tabular}\vspace{0.3cm}

\item \verb+estimagic+

We maintain a \verb+Python+ package that helps to build high-quality and user-friendly implementations of (structural) econometric models. \verb+estimagic+ provides a consistent interface to a large set of global and local optimizers. Complicated optimizations can be monitored using an interactive browser-based dashboard.\vspace{0.3cm}

\begin{tabular}{ll}
\textbf{GitHub}		& \url{OpenSourceEconomics/estimagic}\\
\textbf{Docs}     & \url{estimagic.readthedocs.org}
\end{tabular}\vspace{0.3cm}
\end{itemize}

\noindent Throughout the course, we will make heavy use of \verb+Python+ and its \verb+SciPy+ ecosystem and \verb+Jupyter+ Notebooks. Basic knowledge of this toolchain is a prerequisite. There exist numerous introductory resources, and we provide a curated list at \url{ose-resources.readthedocs.io}. We will use \verb+zulip+ for all course communications, please be sure to join our workspace at \url{ose.zulipchat.com}.

\begin{itemize}

\item \textbf{Introduction}

We provide a general introduction to computational modeling in economics and our OpenSourceEconomics initiative. We outline opportunities for students to get involved.

\begin{itemize}
  \item add citation to OSE website here.
\end{itemize}

\item \textbf{Testing}

We discuss different types of automated tests and how they can be applied to scientific software projects. After presenting the basics, we show some examples in \verb+estimagic+ and \verb+respy+.

\begin{itemize}
  \item \bibentry{Okken.2017}
  \item \bibentry{Arbuckle.2010}
\end{itemize}


\item \textbf{Collaboration}

We introduce continuous integration features of \verb+GitHub+ and discuss workflows for different team sizes.

\begin{itemize}
  \item \bibentry{Chacon.2014}
  \item \bibentry{Bilschak.2016}
\end{itemize}

\item \textbf{Numerical optimization}

After a brief theoretical introduction to local and global optimization, students will solve straightforward optimization problems with \verb+scipy+. We then discuss the limitations of \verb+scipy.optimize+ in typical econometric workflows and introduce the students to the powerful optimization tools of the \verb+estimagic+ package.

\begin{itemize}
  \item \bibentry{Locatelli.2013}
  \item \bibentry{Nocedal.2006}
  \item \bibentry{Gabler.2019}
\end{itemize}

\item \textbf{Numerical derivatives}

After a brief introduction to numerical differentiation, we show how to calculate gradients, Jacobian and Hessian matrices using estimagic.

\begin{itemize}
  \item \bibentry{Brodtkorb.2019}
  \item \bibentry{Ridout.2009}
  \item \bibentry{Gabler.2019}
\end{itemize}

\item \textbf{Numerical integration}

We first introduce the participants to quadrature, Monte-Carlo, and Quasi-Monte-Carlo methods and provide them with (over)simplified guidelines to choose between the different methods. We then show the convergence speed of different approaches in a discrete choice dynamic programming model.

\begin{itemize}
  \item \bibentry{Davis.2007}
  \item \bibentry{Heiss.2008}
  \item \bibentry{Judd.2011}
\end{itemize}

\item \textbf{Discrete choice dynamic programming models}

We first outline the underlying economic, mathematical, and computational model. We then provide an overview of its applications in economics. Finally, we use \verb+respy+ to showcase the application of earlier lessons for numerical integration, parallelization strategies, version control, software testing, and continuous integration.

\begin{itemize}
\item \bibentry{OSE.2020}
\item \bibentry{Keane.2011d}
\item \bibentry{Aguirregabiria.2010}
\end{itemize}

\item \textbf{Execution speed}

Execution speed \verb+Python+ is often called a slow language. Nevertheless, it is one of the most widely used languages for high-performance computing. We teach students how to make their \verb+Python+ code fast with \verb+numpy+ and \verb+numba+ and how to parallelize it using \verb+multiprocessing+ and \verb+joblib+. After the students solve simple examples, we look at some implementations is \verb+respy+.


\begin{itemize}
\item \bibentry{Gorelick.2014}
\item \bibentry{Lanaro.2017}
\end{itemize}


\end{itemize}


\bibliographystyle{apalike}
\nobibliography{../submodules/bibliography/literature}

\end{document}
